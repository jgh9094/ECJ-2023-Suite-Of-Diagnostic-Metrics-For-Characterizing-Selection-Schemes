\PassOptionsToPackage{unicode=true}{hyperref} % options for packages loaded elsewhere
\PassOptionsToPackage{hyphens}{url}
%
\documentclass[]{book}
\usepackage{lmodern}
\usepackage{amssymb,amsmath}
\usepackage{ifxetex,ifluatex}
\usepackage{fixltx2e} % provides \textsubscript
\ifnum 0\ifxetex 1\fi\ifluatex 1\fi=0 % if pdftex
  \usepackage[T1]{fontenc}
  \usepackage[utf8]{inputenc}
  \usepackage{textcomp} % provides euro and other symbols
\else % if luatex or xelatex
  \usepackage{unicode-math}
  \defaultfontfeatures{Ligatures=TeX,Scale=MatchLowercase}
\fi
% use upquote if available, for straight quotes in verbatim environments
\IfFileExists{upquote.sty}{\usepackage{upquote}}{}
% use microtype if available
\IfFileExists{microtype.sty}{%
\usepackage[]{microtype}
\UseMicrotypeSet[protrusion]{basicmath} % disable protrusion for tt fonts
}{}
\IfFileExists{parskip.sty}{%
\usepackage{parskip}
}{% else
\setlength{\parindent}{0pt}
\setlength{\parskip}{6pt plus 2pt minus 1pt}
}
\usepackage{hyperref}
\hypersetup{
            pdftitle={Diagnosing Island Supplemental Material},
            pdfauthor={Jose Guadalupe Hernandez},
            pdfborder={0 0 0},
            breaklinks=true}
\urlstyle{same}  % don't use monospace font for urls
\usepackage{color}
\usepackage{fancyvrb}
\newcommand{\VerbBar}{|}
\newcommand{\VERB}{\Verb[commandchars=\\\{\}]}
\DefineVerbatimEnvironment{Highlighting}{Verbatim}{commandchars=\\\{\}}
% Add ',fontsize=\small' for more characters per line
\usepackage{framed}
\definecolor{shadecolor}{RGB}{248,248,248}
\newenvironment{Shaded}{\begin{snugshade}}{\end{snugshade}}
\newcommand{\AlertTok}[1]{\textcolor[rgb]{0.94,0.16,0.16}{#1}}
\newcommand{\AnnotationTok}[1]{\textcolor[rgb]{0.56,0.35,0.01}{\textbf{\textit{#1}}}}
\newcommand{\AttributeTok}[1]{\textcolor[rgb]{0.77,0.63,0.00}{#1}}
\newcommand{\BaseNTok}[1]{\textcolor[rgb]{0.00,0.00,0.81}{#1}}
\newcommand{\BuiltInTok}[1]{#1}
\newcommand{\CharTok}[1]{\textcolor[rgb]{0.31,0.60,0.02}{#1}}
\newcommand{\CommentTok}[1]{\textcolor[rgb]{0.56,0.35,0.01}{\textit{#1}}}
\newcommand{\CommentVarTok}[1]{\textcolor[rgb]{0.56,0.35,0.01}{\textbf{\textit{#1}}}}
\newcommand{\ConstantTok}[1]{\textcolor[rgb]{0.00,0.00,0.00}{#1}}
\newcommand{\ControlFlowTok}[1]{\textcolor[rgb]{0.13,0.29,0.53}{\textbf{#1}}}
\newcommand{\DataTypeTok}[1]{\textcolor[rgb]{0.13,0.29,0.53}{#1}}
\newcommand{\DecValTok}[1]{\textcolor[rgb]{0.00,0.00,0.81}{#1}}
\newcommand{\DocumentationTok}[1]{\textcolor[rgb]{0.56,0.35,0.01}{\textbf{\textit{#1}}}}
\newcommand{\ErrorTok}[1]{\textcolor[rgb]{0.64,0.00,0.00}{\textbf{#1}}}
\newcommand{\ExtensionTok}[1]{#1}
\newcommand{\FloatTok}[1]{\textcolor[rgb]{0.00,0.00,0.81}{#1}}
\newcommand{\FunctionTok}[1]{\textcolor[rgb]{0.00,0.00,0.00}{#1}}
\newcommand{\ImportTok}[1]{#1}
\newcommand{\InformationTok}[1]{\textcolor[rgb]{0.56,0.35,0.01}{\textbf{\textit{#1}}}}
\newcommand{\KeywordTok}[1]{\textcolor[rgb]{0.13,0.29,0.53}{\textbf{#1}}}
\newcommand{\NormalTok}[1]{#1}
\newcommand{\OperatorTok}[1]{\textcolor[rgb]{0.81,0.36,0.00}{\textbf{#1}}}
\newcommand{\OtherTok}[1]{\textcolor[rgb]{0.56,0.35,0.01}{#1}}
\newcommand{\PreprocessorTok}[1]{\textcolor[rgb]{0.56,0.35,0.01}{\textit{#1}}}
\newcommand{\RegionMarkerTok}[1]{#1}
\newcommand{\SpecialCharTok}[1]{\textcolor[rgb]{0.00,0.00,0.00}{#1}}
\newcommand{\SpecialStringTok}[1]{\textcolor[rgb]{0.31,0.60,0.02}{#1}}
\newcommand{\StringTok}[1]{\textcolor[rgb]{0.31,0.60,0.02}{#1}}
\newcommand{\VariableTok}[1]{\textcolor[rgb]{0.00,0.00,0.00}{#1}}
\newcommand{\VerbatimStringTok}[1]{\textcolor[rgb]{0.31,0.60,0.02}{#1}}
\newcommand{\WarningTok}[1]{\textcolor[rgb]{0.56,0.35,0.01}{\textbf{\textit{#1}}}}
\usepackage{longtable,booktabs}
% Fix footnotes in tables (requires footnote package)
\IfFileExists{footnote.sty}{\usepackage{footnote}\makesavenoteenv{longtable}}{}
\usepackage{graphicx,grffile}
\makeatletter
\def\maxwidth{\ifdim\Gin@nat@width>\linewidth\linewidth\else\Gin@nat@width\fi}
\def\maxheight{\ifdim\Gin@nat@height>\textheight\textheight\else\Gin@nat@height\fi}
\makeatother
% Scale images if necessary, so that they will not overflow the page
% margins by default, and it is still possible to overwrite the defaults
% using explicit options in \includegraphics[width, height, ...]{}
\setkeys{Gin}{width=\maxwidth,height=\maxheight,keepaspectratio}
\setlength{\emergencystretch}{3em}  % prevent overfull lines
\providecommand{\tightlist}{%
  \setlength{\itemsep}{0pt}\setlength{\parskip}{0pt}}
\setcounter{secnumdepth}{5}
% Redefines (sub)paragraphs to behave more like sections
\ifx\paragraph\undefined\else
\let\oldparagraph\paragraph
\renewcommand{\paragraph}[1]{\oldparagraph{#1}\mbox{}}
\fi
\ifx\subparagraph\undefined\else
\let\oldsubparagraph\subparagraph
\renewcommand{\subparagraph}[1]{\oldsubparagraph{#1}\mbox{}}
\fi

% set default figure placement to htbp
\makeatletter
\def\fps@figure{htbp}
\makeatother

\usepackage[]{natbib}
\bibliographystyle{apalike}

\title{Diagnosing Island Supplemental Material}
\author{Jose Guadalupe Hernandez}
\date{2023-06-12}

\begin{document}
\maketitle

{
\setcounter{tocdepth}{1}
\tableofcontents
}
\hypertarget{introduction}{%
\chapter{Introduction}\label{introduction}}

This is the supplemental material for experiments with basic diagnostics.

\hypertarget{about-our-supplemental-material}{%
\section{About our supplemental material}\label{about-our-supplemental-material}}

This supplemental material is hosted on \href{https://github.com}{GitHub} using GitHub pages.
The source code and configuration files used to generate this supplemental material can be found in \href{https://github.com/jgh9094/ECJ-2023-Suite-Of-Diagnostic-Metrics-For-Characterizing-Selection-Schemes}{this GitHub repository}.
We compiled our data analyses and supplemental documentation into this nifty web-accessible book using \href{https://bookdown.org/}{bookdown}.

Our supplemental material includes the following paper figures and statistics:

\begin{itemize}
\tightlist
\item
  Exploitation rate results (Section \ref{exploitation-rate-results})
\item
  Ordered exploitation results (Section \ref{ordered-exploitation-results})
\item
  Contradictory objectives results (Section \ref{contradictory-objectives-results})
\item
  Multi-path exploration results (Section \ref{multi-path-exploration-results})
\end{itemize}

\hypertarget{contributing-authors}{%
\section{Contributing authors}\label{contributing-authors}}

\begin{itemize}
\tightlist
\item
  \href{https://jgh9094.github.io/}{Jose Guadalupe Hernandez}
\item
  \href{https://lalejini.com}{Alexander Lalejini}
\item
  \href{http://ofria.com}{Charles Ofria}
\end{itemize}

\hypertarget{computer-setup}{%
\section{Computer Setup}\label{computer-setup}}

These analyses were conducted in the following computing environment:

\begin{Shaded}
\begin{Highlighting}[]
\KeywordTok{print}\NormalTok{(version)}
\end{Highlighting}
\end{Shaded}

\begin{verbatim}
##                _                           
## platform       x86_64-pc-linux-gnu         
## arch           x86_64                      
## os             linux-gnu                   
## system         x86_64, linux-gnu           
## status                                     
## major          4                           
## minor          3.0                         
## year           2023                        
## month          04                          
## day            21                          
## svn rev        84292                       
## language       R                           
## version.string R version 4.3.0 (2023-04-21)
## nickname       Already Tomorrow
\end{verbatim}

\hypertarget{experimental-setup}{%
\section{Experimental setup}\label{experimental-setup}}

Setting up required variables variables.

\begin{Shaded}
\begin{Highlighting}[]
\CommentTok{# libraries we are using}
\KeywordTok{library}\NormalTok{(ggplot2)}
\KeywordTok{library}\NormalTok{(cowplot)}
\KeywordTok{library}\NormalTok{(dplyr)}
\end{Highlighting}
\end{Shaded}

\begin{verbatim}
## 
## Attaching package: 'dplyr'
\end{verbatim}

\begin{verbatim}
## The following objects are masked from 'package:stats':
## 
##     filter, lag
\end{verbatim}

\begin{verbatim}
## The following objects are masked from 'package:base':
## 
##     intersect, setdiff, setequal, union
\end{verbatim}

\begin{Shaded}
\begin{Highlighting}[]
\KeywordTok{library}\NormalTok{(PupillometryR)}
\end{Highlighting}
\end{Shaded}

\begin{verbatim}
## Loading required package: rlang
\end{verbatim}

\begin{Shaded}
\begin{Highlighting}[]
\CommentTok{# data diractory for gh-pages}
\NormalTok{DATA_DIR =}\StringTok{ '/opt/ECJ-2023-Suite-Of-Diagnostic-Metrics-For-Characterizing-Selection-Schemes/DATA/BASE_DIAGNOSTICS/'}

\CommentTok{# data diractory for local testing}
\CommentTok{# DATA_DIR = 'C:/Users/jgh9094/Desktop/Research/Projects/SelectionDiagnostics/ECJ-2023-Suite-Of-Diagnostic-Metrics-For-Characterizing-Selection-Schemes/DATA/BASE_DIAGNOSTICS/'}

\CommentTok{# graph variables}
\NormalTok{SHAPE =}\StringTok{ }\KeywordTok{c}\NormalTok{(}\DecValTok{5}\NormalTok{,}\DecValTok{3}\NormalTok{,}\DecValTok{1}\NormalTok{,}\DecValTok{2}\NormalTok{,}\DecValTok{6}\NormalTok{,}\DecValTok{0}\NormalTok{,}\DecValTok{4}\NormalTok{,}\DecValTok{20}\NormalTok{,}\DecValTok{1}\NormalTok{)}
\NormalTok{cb_palette <-}\StringTok{ }\KeywordTok{c}\NormalTok{(}\StringTok{'#332288'}\NormalTok{,}\StringTok{'#88CCEE'}\NormalTok{,}\StringTok{'#EE7733'}\NormalTok{,}\StringTok{'#EE3377'}\NormalTok{,}\StringTok{'#117733'}\NormalTok{,}\StringTok{'#882255'}\NormalTok{,}\StringTok{'#44AA99'}\NormalTok{,}\StringTok{'#CCBB44'}\NormalTok{, }\StringTok{'#000000'}\NormalTok{)}
\NormalTok{TSIZE =}\StringTok{ }\DecValTok{26}
\NormalTok{p_theme <-}\StringTok{ }\KeywordTok{theme}\NormalTok{(}
  \DataTypeTok{text =} \KeywordTok{element_text}\NormalTok{(}\DataTypeTok{size =} \DecValTok{28}\NormalTok{),}
  \DataTypeTok{plot.title =} \KeywordTok{element_text}\NormalTok{( }\DataTypeTok{face =} \StringTok{"bold"}\NormalTok{, }\DataTypeTok{size =} \DecValTok{22}\NormalTok{, }\DataTypeTok{hjust=}\FloatTok{0.5}\NormalTok{),}
  \DataTypeTok{panel.border =} \KeywordTok{element_blank}\NormalTok{(),}
  \DataTypeTok{panel.grid.minor =} \KeywordTok{element_blank}\NormalTok{(),}
  \DataTypeTok{legend.title=}\KeywordTok{element_text}\NormalTok{(}\DataTypeTok{size=}\DecValTok{22}\NormalTok{),}
  \DataTypeTok{legend.text=}\KeywordTok{element_text}\NormalTok{(}\DataTypeTok{size=}\DecValTok{23}\NormalTok{),}
  \DataTypeTok{axis.title =} \KeywordTok{element_text}\NormalTok{(}\DataTypeTok{size=}\DecValTok{23}\NormalTok{),}
  \DataTypeTok{axis.text =} \KeywordTok{element_text}\NormalTok{(}\DataTypeTok{size=}\DecValTok{22}\NormalTok{),}
  \DataTypeTok{legend.position=}\StringTok{"bottom"}\NormalTok{,}
  \DataTypeTok{panel.background =} \KeywordTok{element_rect}\NormalTok{(}\DataTypeTok{fill =} \StringTok{"#f1f2f5"}\NormalTok{,}
                                  \DataTypeTok{colour =} \StringTok{"white"}\NormalTok{,}
                                  \DataTypeTok{size =} \FloatTok{0.5}\NormalTok{, }\DataTypeTok{linetype =} \StringTok{"solid"}\NormalTok{)}
\NormalTok{)}
\end{Highlighting}
\end{Shaded}

\begin{verbatim}
## Warning: The `size` argument of `element_rect()` is deprecated as of ggplot2 3.4.0.
## i Please use the `linewidth` argument instead.
## This warning is displayed once every 8 hours.
## Call `lifecycle::last_lifecycle_warnings()` to see where this warning was
## generated.
\end{verbatim}

\begin{Shaded}
\begin{Highlighting}[]
\CommentTok{# default variables}
\NormalTok{REPLICATES =}\StringTok{ }\DecValTok{50}
\NormalTok{DIMENSIONALITY =}\StringTok{ }\DecValTok{100}

\CommentTok{# selection scheme related stuff}
\NormalTok{ACRO =}\StringTok{ }\KeywordTok{c}\NormalTok{(}\StringTok{'tru'}\NormalTok{,}\StringTok{'tor'}\NormalTok{,}\StringTok{'lex'}\NormalTok{,}\StringTok{'gfs'}\NormalTok{,}\StringTok{'pfs'}\NormalTok{,}\StringTok{'nds'}\NormalTok{,}\StringTok{'nov'}\NormalTok{,}\StringTok{'ran'}\NormalTok{)}
\NormalTok{NAMES =}\StringTok{ }\KeywordTok{c}\NormalTok{(}\StringTok{'Truncation (tru)'}\NormalTok{,}\StringTok{'Tournament (tor)'}\NormalTok{,}\StringTok{'Lexicase (lex)'}\NormalTok{, }\StringTok{'Genotypic Fitness Sharing (gfs)'}\NormalTok{,}\StringTok{'Phenotypic Fitness Sharing (pfs)'}\NormalTok{,}\StringTok{'Nondominated Sorting (nds)'}\NormalTok{,}\StringTok{'Novelty Search (nov)'}\NormalTok{,}\StringTok{'Random (ran)'}\NormalTok{)}
\end{Highlighting}
\end{Shaded}

\hypertarget{exploitation-rate-results}{%
\chapter{Exploitation rate results}\label{exploitation-rate-results}}

Here we present the results for \textbf{best performances} found by each selection scheme on the exploitation rate diagnostic.
50 replicates are conducted for each scheme explored.

\hypertarget{analysis-dependencies}{%
\section{Analysis dependencies}\label{analysis-dependencies}}

\begin{Shaded}
\begin{Highlighting}[]
\KeywordTok{library}\NormalTok{(ggplot2)}
\KeywordTok{library}\NormalTok{(cowplot)}
\KeywordTok{library}\NormalTok{(dplyr)}
\KeywordTok{library}\NormalTok{(PupillometryR)}
\end{Highlighting}
\end{Shaded}

\hypertarget{data-setup}{%
\section{Data setup}\label{data-setup}}

\begin{Shaded}
\begin{Highlighting}[]
\CommentTok{# dir = paste(DATA_DIR,'EXPLOITATION_RATE/','over-time.csv', sep = "", collapse = NULL)}
\NormalTok{over_time_df <-}\StringTok{ }\KeywordTok{read.csv}\NormalTok{(}\KeywordTok{paste}\NormalTok{(DATA_DIR,}\StringTok{'EXPLOITATION_RATE/'}\NormalTok{,}\StringTok{'over-time.csv'}\NormalTok{, }\DataTypeTok{sep =} \StringTok{""}\NormalTok{, }\DataTypeTok{collapse =} \OtherTok{NULL}\NormalTok{), }\DataTypeTok{header =} \OtherTok{TRUE}\NormalTok{, }\DataTypeTok{stringsAsFactors =} \OtherTok{FALSE}\NormalTok{)}
\NormalTok{over_time_df}\OperatorTok{$}\NormalTok{scheme <-}\StringTok{ }\KeywordTok{factor}\NormalTok{(over_time_df}\OperatorTok{$}\NormalTok{scheme, }\DataTypeTok{levels =}\NormalTok{ NAMES)}
\end{Highlighting}
\end{Shaded}

\hypertarget{performance-over-time}{%
\section{Performance over time}\label{performance-over-time}}

Best performance in a population over time.

\begin{Shaded}
\begin{Highlighting}[]
\CommentTok{# data for lines and shading on plots}
\NormalTok{lines =}\StringTok{ }\NormalTok{over_time_df }\OperatorTok
\StringTok{  }\KeywordTok{group_by}\NormalTok{(scheme, gen) }\OperatorTok
\StringTok{  }\NormalTok{dplyr}\OperatorTok{::}\KeywordTok{summarise}\NormalTok{(}
    \DataTypeTok{min =} \KeywordTok{min}\NormalTok{(pop_fit_max) }\OperatorTok{/}\StringTok{ }\NormalTok{DIMENSIONALITY,}
    \DataTypeTok{mean =} \KeywordTok{mean}\NormalTok{(pop_fit_max) }\OperatorTok{/}\StringTok{ }\NormalTok{DIMENSIONALITY,}
    \DataTypeTok{max =} \KeywordTok{max}\NormalTok{(pop_fit_max) }\OperatorTok{/}\StringTok{ }\NormalTok{DIMENSIONALITY}
\NormalTok{  )}
\end{Highlighting}
\end{Shaded}

\begin{verbatim}
## `summarise()` has grouped output by 'scheme'. You can override using the
## `.groups` argument.
\end{verbatim}

\begin{Shaded}
\begin{Highlighting}[]
\KeywordTok{ggplot}\NormalTok{(lines, }\KeywordTok{aes}\NormalTok{(}\DataTypeTok{x=}\NormalTok{gen, }\DataTypeTok{y=}\NormalTok{mean, }\DataTypeTok{group =}\NormalTok{ scheme, }\DataTypeTok{fill =}\NormalTok{scheme, }\DataTypeTok{color =}\NormalTok{ scheme, }\DataTypeTok{shape =}\NormalTok{ scheme)) }\OperatorTok{+}
\StringTok{  }\KeywordTok{geom_ribbon}\NormalTok{(}\KeywordTok{aes}\NormalTok{(}\DataTypeTok{ymin =}\NormalTok{ min, }\DataTypeTok{ymax =}\NormalTok{ max), }\DataTypeTok{alpha =} \FloatTok{0.1}\NormalTok{) }\OperatorTok{+}
\StringTok{  }\KeywordTok{geom_line}\NormalTok{(}\DataTypeTok{size =} \FloatTok{0.5}\NormalTok{) }\OperatorTok{+}
\StringTok{  }\KeywordTok{geom_point}\NormalTok{(}\DataTypeTok{data =} \KeywordTok{filter}\NormalTok{(lines, gen }\OperatorTok\StringTok{ }\DecValTok{2000} \OperatorTok{==}\StringTok{ }\DecValTok{0} \OperatorTok{&}\StringTok{ }\NormalTok{gen }\OperatorTok{!=}\StringTok{ }\DecValTok{0}\NormalTok{), }\DataTypeTok{size =} \FloatTok{1.5}\NormalTok{, }\DataTypeTok{stroke =} \FloatTok{2.0}\NormalTok{, }\DataTypeTok{alpha =} \FloatTok{1.0}\NormalTok{) }\OperatorTok{+}
\StringTok{  }\KeywordTok{scale_y_continuous}\NormalTok{(}
    \DataTypeTok{name=}\StringTok{"Average trait score"}\NormalTok{,}
    \DataTypeTok{limits=}\KeywordTok{c}\NormalTok{(}\DecValTok{0}\NormalTok{, }\DecValTok{100}\NormalTok{),}
    \DataTypeTok{breaks=}\KeywordTok{seq}\NormalTok{(}\DecValTok{0}\NormalTok{,}\DecValTok{100}\NormalTok{, }\DecValTok{20}\NormalTok{),}
    \DataTypeTok{labels=}\KeywordTok{c}\NormalTok{(}\StringTok{"0"}\NormalTok{, }\StringTok{"20"}\NormalTok{, }\StringTok{"40"}\NormalTok{, }\StringTok{"60"}\NormalTok{, }\StringTok{"80"}\NormalTok{, }\StringTok{"100"}\NormalTok{)}
\NormalTok{  ) }\OperatorTok{+}
\StringTok{  }\KeywordTok{scale_x_continuous}\NormalTok{(}
    \DataTypeTok{name=}\StringTok{"Generations"}\NormalTok{,}
    \DataTypeTok{limits=}\KeywordTok{c}\NormalTok{(}\DecValTok{0}\NormalTok{, }\DecValTok{50000}\NormalTok{),}
    \DataTypeTok{breaks=}\KeywordTok{c}\NormalTok{(}\DecValTok{0}\NormalTok{, }\DecValTok{10000}\NormalTok{, }\DecValTok{20000}\NormalTok{, }\DecValTok{30000}\NormalTok{, }\DecValTok{40000}\NormalTok{, }\DecValTok{50000}\NormalTok{),}
    \DataTypeTok{labels=}\KeywordTok{c}\NormalTok{(}\StringTok{"0e+4"}\NormalTok{, }\StringTok{"1e+4"}\NormalTok{, }\StringTok{"2e+4"}\NormalTok{, }\StringTok{"3e+4"}\NormalTok{, }\StringTok{"4e+4"}\NormalTok{, }\StringTok{"5e+4"}\NormalTok{)}

\NormalTok{  ) }\OperatorTok{+}
\StringTok{  }\KeywordTok{scale_shape_manual}\NormalTok{(}\DataTypeTok{values=}\NormalTok{SHAPE)}\OperatorTok{+}
\StringTok{  }\KeywordTok{scale_colour_manual}\NormalTok{(}\DataTypeTok{values =}\NormalTok{ cb_palette) }\OperatorTok{+}
\StringTok{  }\KeywordTok{scale_fill_manual}\NormalTok{(}\DataTypeTok{values =}\NormalTok{ cb_palette) }\OperatorTok{+}
\StringTok{  }\KeywordTok{ggtitle}\NormalTok{(}\StringTok{'Performance over time'}\NormalTok{)}\OperatorTok{+}
\StringTok{  }\NormalTok{p_theme }\OperatorTok{+}\StringTok{ }\KeywordTok{theme}\NormalTok{(}\DataTypeTok{legend.title=}\KeywordTok{element_blank}\NormalTok{(),}\DataTypeTok{legend.text=}\KeywordTok{element_text}\NormalTok{(}\DataTypeTok{size=}\DecValTok{12}\NormalTok{)) }\OperatorTok{+}
\StringTok{  }\KeywordTok{guides}\NormalTok{(}
    \DataTypeTok{shape=}\KeywordTok{guide_legend}\NormalTok{(}\DataTypeTok{ncol=}\DecValTok{2}\NormalTok{, }\DataTypeTok{title.position =} \StringTok{"bottom"}\NormalTok{),}
    \DataTypeTok{color=}\KeywordTok{guide_legend}\NormalTok{(}\DataTypeTok{ncol=}\DecValTok{2}\NormalTok{, }\DataTypeTok{title.position =} \StringTok{"bottom"}\NormalTok{),}
    \DataTypeTok{fill=}\KeywordTok{guide_legend}\NormalTok{(}\DataTypeTok{ncol=}\DecValTok{2}\NormalTok{, }\DataTypeTok{title.position =} \StringTok{"bottom"}\NormalTok{)}
\NormalTok{  )}
\end{Highlighting}
\end{Shaded}

\includegraphics{base-diagnostics_files/figure-latex/exp-per-ot-1.pdf}

\hypertarget{ordered-exploitation-results}{%
\chapter{Ordered exploitation results}\label{ordered-exploitation-results}}

Here we present the results for \textbf{best performances} found by each selection scheme on the ordered exploitation diagnostic.
50 replicates are conducted for each scheme explored.

\hypertarget{contradictory-objectives-results}{%
\chapter{Contradictory objectives results}\label{contradictory-objectives-results}}

Here we present the results for \textbf{activation gene coverage} and \textbf{satisfacotory trait coverage} found by each selection scheme on the contradictory objectives diagnostic.
50 replicates are conducted for each scheme explored.

\hypertarget{multi-path-exploration-results}{%
\chapter{Multi-path exploration results}\label{multi-path-exploration-results}}

Here we present the results for \textbf{best performances} found by each selection scheme on the multi-path exploration diagnostic.
50 replicates are conducted for each scheme explored.

\bibliography{book.bib,packages.bib}

\end{document}
